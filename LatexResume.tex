%%%%%%%%%%%%%%%%%%%%%%%%%%%%%%%%%%%%%%%%%
% Medium Length Professional CV
% LaTeX Template
% Version 2.0 (8/5/13)
%
% This template has been downloaded from:
% http://www.LaTeXTemplates.com
%
% Original author:
% Trey Hunner (http://www.treyhunner.com/)
%
% Important note:
% This template requires the resume.cls file to be in the same directory as the
% .tex file. The resume.cls file provides the resume style used for structuring the
% document.
%
%%%%%%%%%%%%%%%%%%%%%%%%%%%%%%%%%%%%%%%%%

%----------------------------------------------------------------------------------------
%	PACKAGES AND OTHER DOCUMENT CONFIGURATIONS
%----------------------------------------------------------------------------------------


\documentclass{resume} % Use the custom resume.cls style

\usepackage[left=0.75in,top=0.6in,right=0.75in,bottom=0.6in]{geometry} % Document margins
\usepackage[utf8]{inputenc}

\name{Gabriel Griggs} % Your name
\address{1118 East Fairview Ave. \\ South Bend, IN 46614} % Your address
\address{} % Your secondary addess (optional)
\address{(574)~$\cdot$~276~$\cdot$~8488 \\ ggriggs@nd.edu} % Your phone number and email
\begin{document}

%----------------------------------------------------------------------------------------
%	EDUCATION SECTION
%----------------------------------------------------------------------------------------

\begin{rSection}{Education}

{\bf University of Notre Dame} \hfill {June 2014} \\ 
B.A. in Applied Mathematics \& the Program of Liberal Studies \\
% \smallskip \\
Overall GPA: 3.81 \\
\textbf{Academic Honors} \\
Dean's List: 6 out of 7 Semesters \hfill Spring 2011 - Fall 2013\\
Student Advisory Council - Program of Liberal Studies \hfill Fall 2012 - Present\\
Elected by students and faculty to present feedback to the department.
\end{rSection}

%----------------------------------------------------------------------------------------
%	WORK EXPERIENCE SECTION
%----------------------------------------------------------------------------------------

\begin{rSection}{Experience}

\begin{rSubsection}{Investment Office, University of Notre Dame}{January 2011 - Present}{IT Intern}{South Bend, IN}
With assets valued at 8 billion dollars, Notre Dame’s Investment Office manages one of the largest educational endowments in the country. Internship is 10-12 hours a week during the school year, full time during the summer.
\item Maintain and develop database, reporting and analytics solutions with the IT and Analytics Director such as Access Reports, Tableau Dashboards and Access Database with SQL Server Backend.
\item Develop Private Equity Report in Access and Dashboard in Tableau.
\item Research and assist in financial modeling, risk modeling and IT workflow solutions.
\item Create Excel / SQL model to mimic various indices.
\item Utilize programming familiarity in: VBA, SQL, MATLAB, C++, Access, Tableau and Excel.
\end{rSubsection}

%------------------------------------------------

\begin{rSubsection}{Finalist - McCloskey Business Plan Competition}{April 2011}{1 of 5 Finalists Selected From 113 Teams}{Mendoza College of Business}
\item Developed Budraps earbud accessories from inception (late 2007) over 3 years with a team of family members.
\item Assisted in: creating an LLC, filing a provisional patent, pursuing an international patent, manufacturing the initial product, creating a website and outsourcing the development of prototype to an engineering firm.
\end{rSubsection}

%------------------------------------------------

\begin{rSubsection}{Men's Rowing}{Fall 2010 - Spring 2013}{3 Year Team Member}{University of Notre Dame}
\item Pursued excellence in a sport that demands sustained year-long intensity, the ability to perform at a high level under pressure and a healthy team culture and atmosphere. Rowing is 15-20+ hours a week, year round.
\item 3rd Best Men’s Club Team in the Country, Top 20 (including varsity teams) \hfill Spring 2011
\end{rSubsection}

\end{rSection}

%----------------------------------------------------------------------------------------
%	TECHNICAL STRENGTHS SECTION
%----------------------------------------------------------------------------------------

\begin{rSection}{Honors}
\textbf{Eagle Scout} \hfill Fall 2009 \\
\textbf{Graduate Award} \hfill Trinity School at Greenlawn - Spring 2010 \\
Presented to a graduate who embodies: discovery of truth, creation of beauty, practice of goodness, rigorous exploration of reality and the free and disciplined exchange of ideas. \\
\textbf{Kay Lewsen Award} \hfill Trinity School at Greenlawn - Spring 2009 \\
Presented to a junior who demonstrates leadership and courage in living out the culture of the school. \\ 
%\begin{tabular}{ @{} >{\bfseries}l @{\hspace{6ex}} l }
%Eagle Scout & Fall 2009\\
%Graduate Award & Spring 2010 \\
%Kay Lewsen Award & Spring 2009 \\
%\end{tabular}
\end{rSection}

%----------------------------------------------------------------------------------------
%	Relevant Coursework
%----------------------------------------------------------------------------------------

\begin{rSection}{Relevant Coursework}
\begin{rSubsection}{Applied and Computational Mathematics}{}{}{}
\textbf{Advanced Scientific Computing}
\item Covering the fundamentals necessary for high performance computing in science and engineering, this course has a specific emphasis on algorithm development,computer implementation and the application of these methods.
\item Specific Languages Used: C/C++, MPI and CUDA
\item Specific Applications: Solving Systems of Linear Equations, Sub-Domain Decomposition for Solving Time-Dependendent Partial Differential Equations on Large Domains, Matrix-Vector Multiplication with CUDA in C using grid/block topology. \\
\textbf{Numerical Analysis}
\item Developing a basic understanding of numerical algorithms and their implementation.
\item Specific Applications: Numerical Solutions to Non-Linear Equations, Interpolation and Polynomial Approximation, Numerical Integration and Differentiation, IVP Problems and Linear Algebra. \\
\textbf{Mathematical Statistics}
\item Topics Include: Random Sampling Distributions, Estimators and Their Properties, Confidence Intervals and Hypothesis Testing, General Linear Model and Analysis of Variance \\
\textbf{Theory of Computing - Spring 2014}
\item The theory of automata and formal languages is developed along with applications. Various classes of automata, formal languages, and the relations between these classes are studied. Restricted models of computation; finite automata and pushdown automata; grammars and their relations to automata; parsing; turing machines; limits of computation; undecidable problems, the classes of P and NP.\\
\textbf{Non-Linear Dynamical Systems - Spring 2014}
\item Linear and nonlinear dynamical systems such as: Duffing’s, Van der Pol’s and Lorentz, Bifurcation Phenomena and Chaos.\\
\textbf{Introduction to Probability} \\
\textbf{Applied Linear Algebra} \\
\textbf{Scientific Computing} \\
\textbf{Calculus Sequence / Differential Equations} \\
\end{rSubsection}
\end{rSection}

\begin{rSection}{Senior Thesis}
\begin{rSubsection}{Program of Liberal Studies}{}{}{}
\item The Senior Thesis in the Program of Liberal Studies is a year-long capstone project with a length requirement of 9,000 - 15,000 words.
\item My thesis is this: The problem of suffering is fundamentally an interpersonal problem and, as such, it cannot be ‘solved’ through analytic philosophy. A much more complete resolution to suffering can be found in the holistic ‘way of life’ presented in Dostoevsky's \emph{Brothers Karamazov} as active love.
\end{rSubsection}
\end{rSection}
%----------------------------------------------------------------------------------------

\end{document}
